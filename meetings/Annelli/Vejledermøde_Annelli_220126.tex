% Options for packages loaded elsewhere
% Options for packages loaded elsewhere
\PassOptionsToPackage{unicode}{hyperref}
\PassOptionsToPackage{hyphens}{url}
\PassOptionsToPackage{dvipsnames,svgnames,x11names}{xcolor}
%
\documentclass[
  10pt,
  letterpaper,
  DIV=11,
  numbers=noendperiod]{scrartcl}
\usepackage{xcolor}
\usepackage[margin=2.5cm]{geometry}
\usepackage{amsmath,amssymb}
\setcounter{secnumdepth}{-\maxdimen} % remove section numbering
\usepackage{iftex}
\ifPDFTeX
  \usepackage[T1]{fontenc}
  \usepackage[utf8]{inputenc}
  \usepackage{textcomp} % provide euro and other symbols
\else % if luatex or xetex
  \usepackage{unicode-math} % this also loads fontspec
  \defaultfontfeatures{Scale=MatchLowercase}
  \defaultfontfeatures[\rmfamily]{Ligatures=TeX,Scale=1}
\fi
\usepackage{lmodern}
\ifPDFTeX\else
  % xetex/luatex font selection
\fi
% Use upquote if available, for straight quotes in verbatim environments
\IfFileExists{upquote.sty}{\usepackage{upquote}}{}
\IfFileExists{microtype.sty}{% use microtype if available
  \usepackage[]{microtype}
  \UseMicrotypeSet[protrusion]{basicmath} % disable protrusion for tt fonts
}{}
\usepackage{setspace}
\makeatletter
\@ifundefined{KOMAClassName}{% if non-KOMA class
  \IfFileExists{parskip.sty}{%
    \usepackage{parskip}
  }{% else
    \setlength{\parindent}{0pt}
    \setlength{\parskip}{6pt plus 2pt minus 1pt}}
}{% if KOMA class
  \KOMAoptions{parskip=half}}
\makeatother
% Make \paragraph and \subparagraph free-standing
\makeatletter
\ifx\paragraph\undefined\else
  \let\oldparagraph\paragraph
  \renewcommand{\paragraph}{
    \@ifstar
      \xxxParagraphStar
      \xxxParagraphNoStar
  }
  \newcommand{\xxxParagraphStar}[1]{\oldparagraph*{#1}\mbox{}}
  \newcommand{\xxxParagraphNoStar}[1]{\oldparagraph{#1}\mbox{}}
\fi
\ifx\subparagraph\undefined\else
  \let\oldsubparagraph\subparagraph
  \renewcommand{\subparagraph}{
    \@ifstar
      \xxxSubParagraphStar
      \xxxSubParagraphNoStar
  }
  \newcommand{\xxxSubParagraphStar}[1]{\oldsubparagraph*{#1}\mbox{}}
  \newcommand{\xxxSubParagraphNoStar}[1]{\oldsubparagraph{#1}\mbox{}}
\fi
\makeatother


\usepackage{longtable,booktabs,array}
\usepackage{calc} % for calculating minipage widths
% Correct order of tables after \paragraph or \subparagraph
\usepackage{etoolbox}
\makeatletter
\patchcmd\longtable{\par}{\if@noskipsec\mbox{}\fi\par}{}{}
\makeatother
% Allow footnotes in longtable head/foot
\IfFileExists{footnotehyper.sty}{\usepackage{footnotehyper}}{\usepackage{footnote}}
\makesavenoteenv{longtable}
\usepackage{graphicx}
\makeatletter
\newsavebox\pandoc@box
\newcommand*\pandocbounded[1]{% scales image to fit in text height/width
  \sbox\pandoc@box{#1}%
  \Gscale@div\@tempa{\textheight}{\dimexpr\ht\pandoc@box+\dp\pandoc@box\relax}%
  \Gscale@div\@tempb{\linewidth}{\wd\pandoc@box}%
  \ifdim\@tempb\p@<\@tempa\p@\let\@tempa\@tempb\fi% select the smaller of both
  \ifdim\@tempa\p@<\p@\scalebox{\@tempa}{\usebox\pandoc@box}%
  \else\usebox{\pandoc@box}%
  \fi%
}
% Set default figure placement to htbp
\def\fps@figure{htbp}
\makeatother





\setlength{\emergencystretch}{3em} % prevent overfull lines

\providecommand{\tightlist}{%
  \setlength{\itemsep}{0pt}\setlength{\parskip}{0pt}}



 


\KOMAoption{captions}{tableheading}
\usepackage{microtype}
\usepackage{ragged2e}
\usepackage{tabularx}
\usepackage{longtable}
\usepackage{booktabs}
\usepackage{array}
\usepackage{pdflscape}
\usepackage{setspace}
\usepackage{float}
\setlength{\emergencystretch}{3em}
\renewcommand{\arraystretch}{1.2}
\makeatletter
\@ifpackageloaded{caption}{}{\usepackage{caption}}
\AtBeginDocument{%
\ifdefined\contentsname
  \renewcommand*\contentsname{Table of contents}
\else
  \newcommand\contentsname{Table of contents}
\fi
\ifdefined\listfigurename
  \renewcommand*\listfigurename{List of Figures}
\else
  \newcommand\listfigurename{List of Figures}
\fi
\ifdefined\listtablename
  \renewcommand*\listtablename{List of Tables}
\else
  \newcommand\listtablename{List of Tables}
\fi
\ifdefined\figurename
  \renewcommand*\figurename{Figure}
\else
  \newcommand\figurename{Figure}
\fi
\ifdefined\tablename
  \renewcommand*\tablename{Table}
\else
  \newcommand\tablename{Table}
\fi
}
\@ifpackageloaded{float}{}{\usepackage{float}}
\floatstyle{ruled}
\@ifundefined{c@chapter}{\newfloat{codelisting}{h}{lop}}{\newfloat{codelisting}{h}{lop}[chapter]}
\floatname{codelisting}{Listing}
\newcommand*\listoflistings{\listof{codelisting}{List of Listings}}
\makeatother
\makeatletter
\makeatother
\makeatletter
\@ifpackageloaded{caption}{}{\usepackage{caption}}
\@ifpackageloaded{subcaption}{}{\usepackage{subcaption}}
\makeatother
\usepackage{bookmark}
\IfFileExists{xurl.sty}{\usepackage{xurl}}{} % add URL line breaks if available
\urlstyle{same}
\hypersetup{
  pdftitle={Vejledermøde\_Annelli\_220126},
  colorlinks=true,
  linkcolor={blue},
  filecolor={Maroon},
  citecolor={Blue},
  urlcolor={Blue},
  pdfcreator={LaTeX via pandoc}}


\title{Vejledermøde\_Annelli\_220126}
\author{}
\date{}
\begin{document}
\maketitle


\setstretch{1.2}
\section{Dagsorden}\label{dagsorden}

\subsection{Tid og sted}\label{tid-og-sted}

\textbf{Tid:} 30 min\\
\textbf{Sted:} Annellis kontor

\subsection{🟡 Intention}\label{intention}

Formålet med mødet er at give en status på ph.d.-projektet,
fondsansøgningerne og drøfte det Cardiovascular Programme og fælles
vejledermøde.

\subsection{🟢 Desired Outcome}\label{desired-outcome}

Ved mødets afslutning har vi:

{[}X{]} \textbf{En fælles opdateret forståelse af projektets status}\\
- \emph{Kurser:} How to manage you phd project Time management
Developing complex interventons Biostat 1. Phd day Welcome to the phd
study

\begin{itemize}
\item
  Oprettet på Danmarks Statistik -\textgreater{} analyser til paper 3
\item
  Afprøvet interviewguide og første interview til paper 2
\item
  Mangler fortsat svar fra SDU vedrørende DBA deling af acc. data til
  Paper 1
\end{itemize}

{[} X{]} **Trygfonden** - Jeg har orienteret mig i Trygfondens fokus og
det synes relevant at søge. - Vi kigger på budgettet og bliver enige om
hvor meget vi skal søge hos Trygfonden

\emph{Beslutning}: Efter at have gennemlæst opslaget, står det klart at
trygfonden ikke støtter rehabiltering som kun foregår i sundhedsvæsnet.
Vi dropper derfor trygfonden. Jeg holder øje med DFF klinisk forskning
og DFF tematiske opslag.

\emph{Fra sidste møde:} Ansøgning til Region Midtjylland er blevet
sorteret fra inden den er blevet læst. Derfor søges den ikke igen. Vi
aftaler at jeg nedprioriterer fondsansøgningerne for en tid. Der skal
søges 3 fonde det kommende år: - Trygfonden (marts) - Jeg ringer og
forhører mig om projektet flugter med deres fokus i år - DFF klinisk
forskning (start juni) - Jeg holder løbende øje med call. Obs på minimum
beløb på ca. 2 millioner. Vi må finde flere poster at skrive på. -
Helsefonden (august) - 300.000 kr.

{[}X{]} \textbf{Truffet beslutning om Cardiovascular Network}\\
- Jeg er i tvivl om hvad jeg får ud af dette netværk. Det virker langt
fra det jeg arbejder med. - Jeg vil gerne prioritere andre Journal Clubs
og valgfag. - Er der noget jeg overser?

\emph{Beslutning:} Jeg afskriver mig Cardiovascular Network og
prioriterer kurser og jounal clubs som padser bedre til mit
forskningsfelt.

{[}X{]} \textbf{Drøfte det fælles vejledermøde d.~19 februar}\\
- Jeg har ikke særlig meget at opdatere på - udskyde det? - Eventuelt en
anden tilgang?

\emph{Beslutning} Vi rykker mødet til slut april. Emner kan være
analyseplan eller foreløbige resultater på artikel 3. Eventuel
publicering af analyseplan. Desuden skal de andre vejederes perspektiv
på analyseplanen for studie 1 og der skal tages stilling til om
analyseplanen skal på som amendment til clinicaltrial.gov.




\end{document}
